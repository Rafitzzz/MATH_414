In this section, we explore how random (you'll see that they're not so random) numbers are generated in our computers and how the programs designed for creating these numbers work.

\begin{definition}
    A random number generatorn (RNG) is a procedure that produces an \textit{infinite} stream of independent, identically distributed (i.i.d.) random variables $U_1, U_2, \ldots \sim \mu$
    accoording to some \textit{probability distribution} $\mu$.
\end{definition}

Recall that a probability dustribution is basically a fancy way of saying “how likely different things are to happen.” Imagine you have a bag of Skittles, and you want to know the chances of grabbing each color. 
A probability distribution is like a chart or rulebook that says: `Red has a 30\% chance, green has 20\%, purple is rare like a unicorn at 5\%, etc.'

\begin{remark}
    Note that an RNG is a \textit{uniform} random number generator if $\mu$ is the \textit{uniform distribution} in $(0,1)$. Where the uniform distribution is the “everyone gets a fair share” version of probability. Every outcome has the same chance of happening.
\end{remark}

Since all of the current RNG's are based on algorithms, they produce a \textit{purely deterministic} (i.e. the outcome is already locked in once you know the rules) stream of variables $U_1, U_2, U_3, \ldots$ which `look like' a stream of i.i.d. random variables. 
For this reason, algorithmic generators are called pseudo-random number generators.